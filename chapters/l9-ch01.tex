% l9-ch01.tex
% פרק 1: מבוא והשקפה וקטורית
% מחבר: ד"ר יורם סגל
% תאריך: ספטמבר 2025

\hebrewsection{מבוא: עולמו הנסתר של הווקטור}

\noindent\textbf{\en{Introduction: The Hidden World of the Vector}}

\vspace{0.5cm}

בפרק זה נחקור את הבסיס הפילוסופי והמתמטי של הייצוג הווקטורי בבינה מלאכותית. נראה כיצד כל יחידת מידע – תמונה, מילה, או חולה במרפאה – הופכת לנקודה במרחב רב-ממדי, וכיצד מושג זה מהווה את היסוד של כל \en{AI} מודרני.

\hebrewsubsection{מבט ראשון: מה רואים כשמביטים בתמונה?}

כשאתם מביטים בתמונה של חתול, מה המוח שלכם באמת רואה? האם הוא רואה יצור פרוותי חמוד עם שפם? או שמא הוא רואה משהו עמוק יותר – רשת מורכבת של \num{1000000} נקודות אור, כל אחת מהן נושאת מספר המייצג עוצמת בהירות?

זוהי השאלה המרכזית שמניעה את כל תחום הבינה המלאכותית המודרני: \textbf{כיצד ניתן להפוך מידע לווקטור, ואיך ווקטורים הופכים למשמעות?}

\hebrewsubsection{דקארט והמצאת השפה המספרית של המרחב}

ב\en{-}\hebyear{1637}, כשרנה דקארט \en{(René Descartes)} פרסם את ספרו \en{"La Géométrie"}, הוא לא יכול היה לדמיין שהרעיון הפשוט שלו – לייצג נקודה במרחב באמצעות מספרים – יהפוך ליסוד של מהפכה טכנולוגית שתגיע \num{400} שנה מאוחר יותר \cite{strang2019}.

דקארט המציא את מערכת הצירים הקרטזית, ועמה את הרעיון המהפכני שכל נקודה במרחב ניתנת לתיאור מספרי. אך רק במאה ה\en{-}\num{20}, כשמדעני המחשב החלו לחשוב על דרכים לייצג מידע, הבינו שהרעיון של דקארט הוא לא רק כלי גיאומטרי – \textbf{הוא שפה אוניברסלית לתיאור המציאות}.

\textbf{שאלה למחשבה:} אם דקארט יכול היה לייצג נקודה במרחב תלת-ממדי עם \num{3} מספרים, כמה מספרים נדרשים כדי לייצג תמונת חתול?

\hebrewsubsection{הווקטור: לא רק חץ במרחב}

בקורסים מתמטיים מסורתיים, וקטור מוגדר לרוב כ\en{"חץ במרחב"} – יש לו כיוון ואורך. אך בעולם מדעי הנתונים ולמידת המכונה, הווקטור הוא משהו עמוק יותר בהרבה: \textbf{הוא ייצוג של ישות במרחב מופשט} \cite{strang2019}.

\textbf{דוגמה מהחיים – חולה במרפאה:}

אם נתון לכם חולה במרפאה, איך תתארו אותו כווקטור? התשובה: כל תכונה נמדדת של החולה – גיל, משקל, לחץ דם, רמת סוכר, טמפרטורה – הופכת ל\en{-}component אחד בווקטור. אם יש לנו \num{50} תכונות נמדדות, החולה מיוצג כנקודה במרחב בעל \num{50} ממדים:

\[
\vec{p}_{\text{patient}} = \begin{bmatrix}
\text{age} \\
\text{weight} \\
\text{blood\_pressure} \\
\text{glucose} \\
\vdots \\
\text{feature}_{50}
\end{bmatrix} \in \mathbb{R}^{50}
\]

הסימון $\mathbb{R}^{50}$ משמעותו \en{"מרחב אוקלידי של \num{50} ממדים"} – מרחב שבו כל נקודה מוגדרת על ידי \num{50} מספרים ממשיים. זהו המרחב שבו \textbf{כל החולים שלנו חיים}, במובן מתמטי.

\hebrewsubsection{מתמונת חתול לווקטור: תהליך הוקטוריזציה}

כפי שהדגיש ד"ר יורם סגל בהרצאתו, תמונה של $\num{1000} \times \num{1000}$ פיקסלים בצבע \en{RGB} מכילה $\num{3} \times \num{10}^6$ ערכים מספריים (כל פיקסל מיוצג על ידי \num{3} ערכים: אדום, ירוק, כחול).

\textbf{המתמטיקה מאחורי הוקטוריזציה:}

כדי להפוך אותה לייצוג וקטורי הניתן לניתוח על ידי מודל למידת מכונה, אנו מבצעים תהליך של \en{"Flattening"} (השטחה). המטריצה המבנית מפורקת לרצף של מספרים, היוצר וקטור באורך של שלושה מיליון.

אם $\mathbf{I} \in \mathbb{R}^{h \times w \times c}$ היא תמונה בגובה $h$, רוחב $w$ ו\en{-}$c$ ערוצי צבע, אזי הוקטוריזציה מוגדרת כ:

\begin{equation}
\text{vec}(\mathbf{I}) = \mathbf{v} \in \mathbb{R}^{h \cdot w \cdot c}
\end{equation}

כאשר $\mathbf{v} = [I_{1,1,1}, I_{1,1,2}, \ldots, I_{h,w,c}]^T$ – רצף של כל הפיקסלים בזה אחר זה.

\textbf{משמעות עמוקה:} תמונת החתול הפכה לנקודה \textit{אחת ויחידה} במרחב בעל שלושה מיליון ממדים. כל תמונה שונה במעט תהיה נקודה אחרת באותו מרחב עצום.

\hebrewsubsection{מכפלה סקלרית: מדד הדמיון האולטימטיבי}

אחת הפעולות החשובות ביותר על וקטורים היא \textbf{מכפלה סקלרית} \en{(Dot Product, Inner Product)}. היא מוגדרת כך:

\textbf{הגדרה \num{1.1} – מכפלה סקלרית (אלגברית):}
\begin{equation}
\langle \vec{u}, \vec{v} \rangle = \sum_{i=1}^{n} u_i v_i = u_1 v_1 + u_2 v_2 + \cdots + u_n v_n
\end{equation}

\textbf{שאלה מהותית שהעלה המרצה:} מה משמעות המכפלה הסקלרית? למה היא כל כך חשובה?

התשובה מגיעה מנוסחה חלופית, גיאומטרית, שגילה המתמטיקאי הגרמני \en{Hermann von Helmholtz} במאה ה\en{-}\num{19}:

\begin{equation}
\langle \vec{u}, \vec{v} \rangle = \|\vec{u}\| \cdot \|\vec{v}\| \cdot \cos(\theta)
\end{equation}

כאשר $\theta$ היא הזווית בין שני הווקטורים, ו\en{-}$\|\vec{u}\|$ היא \textbf{הנורמה} \en{(Norm)} או אורך הווקטור:

\begin{equation}
\|\vec{u}\| = \sqrt{u_1^2 + u_2^2 + \cdots + u_n^2} = \sqrt{\sum_{i=1}^{n} u_i^2}
\end{equation}

\hebrewsubsection{הוכחה: שתי ההגדרות זהות}

\textbf{משפט \num{1.1}:} ההגדרה האלגברית (משוואה \num{1.2}) וההגדרה הגיאומטרית (משוואה \num{1.3}) של מכפלה סקלרית שוות.

\textbf{הוכחה:}

\textit{צעד 1: משפט הקוסינוסים.}

במשולש עם צלעות $\|\vec{u}\|$, $\|\vec{v}\|$ וזווית $\theta$ ביניהן, הצלע השלישית היא $\|\vec{u} - \vec{v}\|$. לפי משפט הקוסינוסים:

\[
\|\vec{u} - \vec{v}\|^2 = \|\vec{u}\|^2 + \|\vec{v}\|^2 - \num{2}\|\vec{u}\|\|\vec{v}\|\cos(\theta)
\]

\textit{צעד 2: פיתוח אלגברי של האגף השמאלי.}

נפתח את $\|\vec{u} - \vec{v}\|^2$ באמצעות ההגדרה הא לגברית:

\begin{align}
\|\vec{u} - \vec{v}\|^2 &= \sum_{i=1}^{n}(u_i - v_i)^2 \nonumber \\
&= \sum_{i=1}^{n}(u_i^2 - \num{2}u_i v_i + v_i^2) \nonumber \\
&= \sum_{i=1}^{n} u_i^2 + \sum_{i=1}^{n} v_i^2 - \num{2}\sum_{i=1}^{n} u_i v_i \nonumber \\
&= \|\vec{u}\|^2 + \|\vec{v}\|^2 - \num{2}\langle \vec{u}, \vec{v} \rangle
\end{align}

\textit{צעד 3: השוואת הביטויים.}

מצעד \num{1}: $\|\vec{u} - \vec{v}\|^2 = \|\vec{u}\|^2 + \|\vec{v}\|^2 - \num{2}\|\vec{u}\|\|\vec{v}\|\cos(\theta)$

מצעד \num{2}: $\|\vec{u} - \vec{v}\|^2 = \|\vec{u}\|^2 + \|\vec{v}\|^2 - \num{2}\langle \vec{u}, \vec{v} \rangle$

ביטול $\|\vec{u}\|^2 + \|\vec{v}\|^2$ משני האגפים מוביל ל:

\[
\num{2}\langle \vec{u}, \vec{v} \rangle = \num{2}\|\vec{u}\| \cdot \|\vec{v}\| \cdot \cos(\theta)
\]

חלוקה ב\en{-}\num{2} נותנת:

\[
\langle \vec{u}, \vec{v} \rangle = \|\vec{u}\| \cdot \|\vec{v}\| \cdot \cos(\theta) \quad \blacksquare
\]

\textbf{משמעות עמוקה:} מכפלה סקלרית היא \textbf{מדד דמיון גיאומטרי}.

\begin{itemize}
\item אם $\theta = \num{0}°$ (וקטורים באותו כיוון): $\cos(\theta) = \num{1}$ והמכפלה מקסימלית → \textbf{דמיון מלא}
\item אם $\theta = \num{90}°$ (וקטורים אורתוגונליים): $\cos(\theta) = \num{0}$ והמכפלה מתאפסת → \textbf{אין דמיון כלל}
\item אם $\theta = \num{180}°$ (וקטורים מנוגדים): $\cos(\theta) = \num{-1}$ והמכפלה מינימלית → \textbf{ניגוד מלא}
\end{itemize}

\hebrewsubsection{קשר ל\en{-}AI: \en{Cosine Similarity} ומנועי חיפוש}

מנועי חיפוש מודרניים כמו \en{Google} ומערכות המלצה כמו \en{Netflix} משתמשים בווריאציה על מכפלה סקלרית שנקראת \textbf{\en{Cosine Similarity}} – דמיון קוסינוס:

\begin{equation}
\text{similarity}(\vec{u}, \vec{v}) = \frac{\langle \vec{u}, \vec{v} \rangle}{\|\vec{u}\| \cdot \|\vec{v}\|} = \cos(\theta)
\end{equation}

חישוב זה \textbf{מנרמל} את המכפלה הסקלרית כך שהתוצאה תמיד בין $\num{-1}$ ל\en{-}$\num{+1}$, ללא תלות באורכי הווקטורים.

\textbf{למה נורמליזציה חשובה?}

בלי נורמליזציה, וקטור ארוך יקבל מכפלה סקלרית גדולה יותר רק בגלל האורך, לא בגלל הדמיון. \en{Cosine Similarity} פותר זאת על ידי מדידת \textit{הזווית בלבד}.

\textbf{דוגמה מהחיים – מנוע חיפוש:}

כשאתם מקלידים שאילתה \en{"machine learning tutorial"}, מנוע החיפוש:

\begin{enumerate}
\item \textbf{ממיר את השאילתה לווקטור} $\vec{q}$: כל מילה מקבלת משקל (למשל, באמצעות \en{TF-IDF} או \en{Word2Vec} \cite{mikolov2013})
\item \textbf{משווה את} $\vec{q}$ \textbf{לכל מסמך} $\vec{d}_i$ במאגר באמצעות \en{Cosine Similarity}
\item \textbf{מחזיר את המסמכים} עם הדמיון הגבוה ביותר (ציון קוסינוס הכי קרוב ל\en{-}\num{1})
\end{enumerate}

\textbf{פונקציות \en{NumPy} למכפלה סקלרית ונורמה:}

\begin{hebrewtable}[H]
\caption{פונקציות \en{NumPy} למכפלה סקלרית, נורמה ו\en{-}Cosine Similarity}
\centering
\begin{rtltabular}{|l|l|p{5.5cm}|}
\hline
\textbf{\en{Function}} & \textbf{\hebcell{תפקיד}} & \textbf{\hebcell{שימוש והסבר}} \\
\hline
\en{np.dot(u, v)} & \hebcell{מכפלה סקלרית} & \hebcell{מחשבת $\sum u_i v_i$ – הליבה של כל חישוב דמיון} \\
\hline
\en{np.linalg.norm(u)} & \hebcell{נורמה \en{(L2)}} & \hebcell{מחשבת $\sqrt{\sum u_i^2}$ – אורך הווקטור} \\
\hline
\en{@} (אופרטור) & \hebcell{מכפלת מטריצות} & \hebcell{תחביר קצר: \en{u @ v} זהה ל\en{-}np.dot(u, v)} \\
\hline
\end{rtltabular}
\end{hebrewtable}

\textbf{הערה חשובה:} פונקציות אלה הן הבסיס של כל אלגוריתם \en{AI}. נשתמש בהן שוב ושוב לאורך הספר.

\textbf{פסאודו-קוד – \en{Cosine Similarity}:}

\begin{pythonbox}[\en{Cosine Similarity} – מימוש ממוקד]
import numpy as np

def cosine_similarity(u, v):
    """
    מחשבת Cosine Similarity בין שני וקטורים.
    
    Args:
        u, v: וקטורי NumPy בעלי אותו אורך
        
    Returns:
        float: ציון דמיון בין -1 ל-1
    """
    # מכפלה סקלרית - מונה
    dot_product = np.dot(u, v)
    
    # נורמות - מכנה
    norm_u = np.linalg.norm(u)
    norm_v = np.linalg.norm(v)
    
    # Cosine Similarity
    return dot_product / (norm_u * norm_v)

# דוגמה: שני מסמכים (ווקטורי TF-IDF מדומים)
doc1 = np.array([1, 2, 3, 0, 0])
doc2 = np.array([1, 1, 0, 4, 5])

print(f"Similarity: {cosine_similarity(doc1, doc2):.3f}")
# תוצאה: 0.277 - דמיון חלש
\end{pythonbox}

\hebrewsubsection{\en{Word Embeddings}: המהפכה של \en{Word2Vec}}

אחת ההצלחות המרשימות ביותר של ייצוג וקטורי היא \textbf{\en{Word2Vec}}, שפותחה על ידי \en{Tomas Mikolov} ועמיתיו ב\en{-}Google ב\en{-}\hebyear{2013} \cite{mikolov2013}.

\textbf{הרעיון המהפכני:} לייצג כל מילה בשפה כווקטור בן $\sim\num{300}$ ממדים, כך ש\textbf{מילים דומות במשמעות יהיו קרובות במרחב הווקטורי}.

\textbf{הפלא המתמטי של \en{Word2Vec}:}

אחד הממצאים המפורסמים ביותר הוא שחישובים אריתמטיים על וקטורי מילים מניבים תוצאות משמעותיות סמנטית:

\[
\vec{\text{king}} - \vec{\text{man}} + \vec{\text{woman}} \approx \vec{\text{queen}}
\]

\textbf{שאלה שמעלה פלא:} איך זה אפשרי? איך חיבור וחיסור של מספרים יכול לבטא יחסים סמנטיים?

\textbf{התשובה:} הווקטורים לומדים \textbf{יחסים}. הווקטור $\vec{\text{king}} - \vec{\text{man}}$ מייצג את המושג המופשט \en{"מלכות זכרית"}, וכשמוסיפים $\vec{\text{woman}}$ מקבלים \en{"מלכות נשית"} = $\vec{\text{queen}}$.

זה עובד בגלל \textbf{הנחת הדיסטריבוציה} \en{(Distributional Hypothesis)} שניסח הבלשן \en{Zellig Harris} ב\en{-}\hebyear{1954} \cite{harris1954}:

\begin{quote}
\textit{\en{"Words that occur in similar contexts tend to have similar meanings"}}

"מילים שמופיעות בהקשרים דומים נוטות לשאת משמעות דומה"
\end{quote}

אלגוריתם \en{Word2Vec} לומד וקטורים שמנבאים את הקונטקסט של מילה, ובכך \textbf{מקודדים משמעות סמנטית כמרחק גיאומטרי}.

\hebrewsubsection{תרגיל תכנות עצמי \num{1.1} – חיפוש מילים דומות}

\textbf{מטרה:} להבין איך \en{Cosine Similarity} משמש למציאת מילים דומות במודל \en{Word2Vec}.

\textbf{משימה:}
\begin{enumerate}
\item טענו וקטורי \en{Word2Vec} מוכנים (למשל, מ\en{-}Gensim או מודל \en{GloVe})
\item בחרו מילה (למשל, \en{"computer"})
\item חשבו \en{Cosine Similarity} בינה לבין כל המילים האחרות במאגר
\item הציגו את \num{10} המילים הדומות ביותר
\end{enumerate}

\textbf{פסאודו-קוד:}

\begin{pythonbox}[חיפוש מילים דומות ב\en{-}Word2Vec]
from gensim.models import KeyedVectors

# טעינת מודל Word2Vec מוכן (להוריד מהאינטרנט)
model = KeyedVectors.load_word2vec_format(
    'GoogleNews-vectors-negative300.bin', 
    binary=True
)

# מציאת מילים דומות - הפונקציה משתמשת ב-Cosine Similarity
similar_words = model.most_similar('computer', topn=10)

# הדפסה
print("המילים הדומות ביותר ל-'computer':")
for word, similarity in similar_words:
    print(f"  {word}: {similarity:.3f}")
\end{pythonbox}

\textbf{תוצאה צפויה:}
\begin{itemize}
\item \en{laptop}: \num{0.82}
\item \en{software}: \num{0.78}
\item \en{technology}: \num{0.75}
\item \en{hardware}: \num{0.72}
\item \en{PC}: \num{0.70}
\end{itemize}

\hebrewsubsection{אזהרה: הסכנה בווקטורים – \en{Bias} ודעות קדומות}

ב\en{-}\hebyear{2016}, חוקרים מאוניברסיטת \en{Princeton} גילו תופעה מדאיגה: מודלי \en{Word2Vec} שאומנו על טקסטים מהאינטרנט הטמיעו \textbf{דעות קדומות חברתיות} \cite{caliskan2017}.

\textbf{דוגמאות להטיות שנמצאו:}

\begin{itemize}
\item $\vec{\text{man}} : \vec{\text{programmer}}$ כמו $\vec{\text{woman}} : \vec{\text{homemaker}}$
\item $\vec{\text{man}} : \vec{\text{doctor}}$ כמו $\vec{\text{woman}} : \vec{\text{nurse}}$
\item קשרים חזקים יותר בין שמות אירופאים למילים חיוביות לעומת שמות אפרו-אמריקאים
\end{itemize}

המודל למד שגברים קשורים למקצועות טכנולוגיים, ונשים למקצועות ביתיים – לא משום שזו אמת, אלא כי \textbf{כך התייחסו הטקסטים שממנו למד}.

\textbf{שאלה מוסרית שהעלה ד"ר סגל:} אם \en{AI} לומד מהעולם, והעולם מוטה – האם \en{AI} מחזק את המוטיות?

\textbf{התשובה המחקרית:} כן, אלא אם כן מתערבים באופן פעיל. זו אחת המשימות המרכזיות של תחום \textbf{\en{Fairness in AI}}. חוקרים כמו \en{Tolga Bolukbasi} ועמיתיו פיתחו שיטות ל\textbf{ניטרול הטיות} \en{(Debiasing)} בווקטורי מילים \cite{bolukbasi2016}.

\textbf{שיטת הניטרול:}

השיטה מבוססת על זיהוי \textbf{כיווני הטיה} במרחב הווקטורי. למשל, הכיוון $\vec{d}_{\text{gender}} = \vec{\text{he}} - \vec{\text{she}}$ מייצג את ממד המגדר. לאחר מכן, מסירים את המרכיב הזה ממילים שאמורות להיות ניטרליות:

\[
\vec{\text{doctor}}_{\text{debiased}} = \vec{\text{doctor}} - \text{proj}_{\vec{d}_{\text{gender}}}(\vec{\text{doctor}})
\]

כאשר $\text{proj}_{\vec{d}}(\vec{v})$ היא ההטלה של $\vec{v}$ על הכיוון $\vec{d}$.

\textbf{מסקנה:} ייצוג וקטורי הוא כלי עוצמתי, אך הוא משקף את הנתונים שממנו למד. \textbf{אחריות החוקר היא להבטיח שהמודל לא מנציח אפליה}.

\hebrewsubsection{מרחבים וקטוריים: הפורמליזם המתמטי}

עד כה דיברנו על וקטורים בצורה אינטואיטיבית. אך מה באמת הופך אוסף של מספרים לווקטור? התשובה מגיעה מ\textbf{תורת המרחבים הווקטוריים} \en{(Vector Space Theory)}, שפותחה במאה ה\en{-}\num{19} על ידי מתמטיקאים כמו \en{Hermann Grassmann} (\hebyear{1844}) ו\en{-}Giuseppe Peano (\hebyear{1888}).

\textbf{הגדרה \num{1.2} – מרחב וקטורי:}

מרחב וקטורי $V$ מעל שדה $\mathbb{F}$ (בדרך כלל $\mathbb{R}$ או $\mathbb{C}$) הוא קבוצה עם שתי פעולות:

\begin{enumerate}
\item \textbf{חיבור וקטורים:} $\vec{u} + \vec{v} \in V$ לכל $\vec{u}, \vec{v} \in V$
\item \textbf{כפל בסקלר:} $\alpha \vec{u} \in V$ לכל $\alpha \in \mathbb{F}$ ו\en{-}$\vec{u} \in V$
\end{enumerate}

שמקיימות \num{8} אקסיומות:

\begin{hebrewtable}[H]
\caption{אקסיומות מרחב וקטורי}
\centering
\begin{rtltabular}{|r|p{10cm}|}
\hline
\textbf{\hebcell{אקסיומה}} & \textbf{\hebcell{תכונה}} \\
\hline
\num{1} & \hebcell{קומוטטיביות: $\vec{u} + \vec{v} = \vec{v} + \vec{u}$} \\
\hline
\num{2} & \hebcell{אסוציאטיביות: $(\vec{u} + \vec{v}) + \vec{w} = \vec{u} + (\vec{v} + \vec{w})$} \\
\hline
\num{3} & \hebcell{קיום איבר אפס: קיים $\vec{0}$ כך ש\en{-}$\vec{u} + \vec{0} = \vec{u}$} \\
\hline
\num{4} & \hebcell{קיום איבר נגדי: לכל $\vec{u}$ קיים $-\vec{u}$ כך ש\en{-}$\vec{u} + (-\vec{u}) = \vec{0}$} \\
\hline
\num{5} & \hebcell{כפל באחד: $\num{1} \cdot \vec{u} = \vec{u}$} \\
\hline
\num{6} & \hebcell{דיסטריבוטיביות: $\alpha(\vec{u} + \vec{v}) = \alpha\vec{u} + \alpha\vec{v}$} \\
\hline
\num{7} & \hebcell{דיסטריבוטיביות: $(\alpha + \beta)\vec{u} = \alpha\vec{u} + \beta\vec{u}$} \\
\hline
\num{8} & \hebcell{אסוציאטיביות כפל: $(\alpha\beta)\vec{u} = \alpha(\beta\vec{u})$} \\
\hline
\end{rtltabular}
\end{hebrewtable}

\textbf{למה זה חשוב ל\en{-}AI?}

ברגע שהוכחנו שאוסף של ישויות (תמונות, מסמכים, חולים) יוצר מרחב וקטורי, אנחנו יכולים להשתמש ב\textbf{כל הכלים של אלגברה ליניארית}:

\begin{itemize}
\item פירוק \en{SVD (Singular Value Decomposition)}
\item ערכים עצמיים \en{(Eigenvalues)} ווקטורים עצמיים \en{(Eigenvectors)}
\item הטלות \en{(Projections)} והטרנספורמציות ליניאריות
\item \en{PCA (Principal Component Analysis)} להפחתת ממדיות
\end{itemize}

כל אלו הם כלים שנשתמש בהם לאורך הספר לניתוח ועיבוד נתונים.

\hebrewsubsection{המעבר למימד גבוה: ברכה או קללה?}

ב\en{-}\hebyear{1961}, הסטטיסטיקאי והמתמטיקאי \en{Richard Bellman} טבע את המונח המפורסם \textbf{\en{"Curse of Dimensionality"}} – קללת המימדיות \cite{bellman1957}.

\textbf{הבעיה המתמטית:}

\en{Bellman} הבין שככל שמוסיפים תכונות (ממדים) לנתונים, \textbf{נפח המרחב גדל באופן אקספוננציאלי}, והנתונים הופכים ספרסיים \en{(Sparse)} יותר ויותר.

\textbf{דוגמה מספרית שהדגים המרצה:}

נניח שאנחנו רוצים לדגום \num{10} נקודות לאורך כל ממד כדי לכסות את המרחב בצפיפות סבירה:

\begin{itemize}
\item ב\en{-}\num{1} ממד: $\num{10}^1 = \num{10}$ נקודות
\item ב\en{-}\num{2} ממדים: $\num{10}^2 = \num{100}$ נקודות
\item ב\en{-}\num{3} ממדים: $\num{10}^3 = \num{1000}$ נקודות
\item ב\en{-}\num{10} ממדים: $\num{10}^{10} = \num{10000000000}$ נקודות (עשרה מיליארד!)
\item ב\en{-}\num{100} ממדים: $\num{10}^{100}$ – מספר גדול יותר מכמות האטומים ביקום!
\end{itemize}

\textbf{שאלת המפתח שהעלה ד"ר סגל:}

אם יש לנו תמונת חתול $\num{1000} \times \num{1000}$ פיקסלים (מיליון תכונות), כמה דוגמאות אימון נדרשות כדי לאמן מודל \en{Brute Force}?

\textbf{התשובה המתמטית:}

לפי כלל האצבע שנידון בהרצאה:
\begin{itemize}
\item \textbf{למידת מכונה קלאסית:} מינימום \num{30} דוגמאות לכל תכונה
\item \textbf{למידה עמוקה:} מינימום \num{100} דוגמאות לכל תכונה
\end{itemize}

עבור מיליון תכונות: $\num{100} \times \num{10}^6 = \num{100000000}$ דוגמאות (מאה מיליון!)

\textbf{הפתרון – רשתות \en{CNN}:}

זו הסיבה שרשתות \en{Convolutional Neural Networks (CNN)} חיוניות \cite{lecun1998, krizhevsky2012}. הן מפחיתות באופן דרמטי את מספר הפרמטרים באמצעות:

\begin{enumerate}
\item \textbf{שיתוף משקלים} \en{(Weight Sharing)}: אותו פילטר משמש בכל התמונה
\item \textbf{קישוריות מקומית} \en{(Local Connectivity)}: כל נוירון מחובר רק לאזור קטן
\item \textbf{הירארכיה של תכונות:} למידה הדרגתית מתכונות פשוטות (קצוות) למורכבות (פנים)
\end{enumerate}

\textbf{הישג מפורסם:} רשת \en{AlexNet} של \en{Krizhevsky, Sutskever} ו\en{-}Hinton (\hebyear{2012}) אימנה על \num{1.2} מיליון תמונות עם $\sim\num{60}$ מיליון פרמטרים, והשיגה פריצת דרך בסיווג תמונות \cite{krizhevsky2012}.

זו תהיה נקודת המוצא לפרק הבא, שבו נצלול לעומקה של קללת המימדיות ונראה כיצד היא משפיעה על כל אלגוריתם למידה.

\hebrewsubsection{תרגיל תכנות עצמי \num{1.2} – השוואת מרחקים בממדים שונים}

\textbf{מטרה:} להמחיש את קללת המימדיות באופן מעשי.

\textbf{משימה:}

\begin{enumerate}
\item צרו \num{100} נקודות אקראיות במרחבים בממדים שונים: \num{2}, \num{10}, \num{100}, \num{1000}
\item חשבו את המרחק האוקלידי בין כל זוג נקודות
\item חשבו את ממוצע המרחקים ואת סטיית התקן
\item הציגו גרף: ציר X = מימד, ציר Y = יחס סטיית תקן/ממוצע
\end{enumerate}

\textbf{פסאודו-קוד:}

\begin{pythonbox}[המחשת קללת המימדיות]
import numpy as np
import matplotlib.pyplot as plt

def curse_of_dimensionality_demo(dims, n_points=100):
    """
    ממחיש את קללת המימדיות על ידי חישוב מרחקים.
    """
    results = []
    
    for d in dims:
        # יצירת נקודות אקראיות
        points = np.random.rand(n_points, d)
        
        # חישוב כל המרחקים
        distances = []
        for i in range(n_points):
            for j in range(i+1, n_points):
                dist = np.linalg.norm(points[i] - points[j])
                distances.append(dist)
        
        # סטטיסטיקות
        mean_dist = np.mean(distances)
        std_dist = np.std(distances)
        ratio = std_dist / mean_dist  # כשזה שואף ל-0, הקללה חזקה
        
        results.append(ratio)
        print(f"Dim={d}: Mean={mean_dist:.3f}, Std={std_dist:.3f}, Ratio={ratio:.3f}")
    
    return results

# הרצה
dimensions = [2, 5, 10, 50, 100, 500, 1000]
ratios = curse_of_dimensionality_demo(dimensions)

# גרף
plt.plot(dimensions, ratios, marker='o')
plt.xlabel('Number of Dimensions')
plt.ylabel('Std/Mean Ratio')
plt.title('Curse of Dimensionality')
plt.grid(True)
plt.show()
\end{pythonbox}

\textbf{תוצאה צפויה:} היחס שואף ל\en{-}\num{0} ככל שהממד גדל – כל הנקודות נעשות "באותו מרחק" זו מזו, ומדדי מרחק מאבדים משמעות.

\hebrewsubsection{סיכום ומבט קדימה}

\textbf{מה למדנו בפרק זה?}

\begin{enumerate}
\item \textbf{וקטור הוא ייצוג מופשט} – לא רק חץ, אלא כל ישות הניתנת לתיאור מספרי (תמונה, מילה, חולה)
\item \textbf{מכפלה סקלרית היא מדד דמיון גיאומטרי} – הבסיס לחיפוש, המלצות, ו\en{-}NLP
\item \textbf{\en{Cosine Similarity} הוא הכלי המרכזי} – מנרמל מרחקים ומודד זווית בלבד
\item \textbf{\en{Word2Vec} הוא דוגמה מבריקה} – מילים הופכות לווקטורים והסמנטיקה הופכת לגיאומטריה
\item \textbf{הטיות ב\en{-}AI הן בעיה אמיתית} – מודלים לומדים מהעולם, כולל דעות קדומות
\item \textbf{מרחבים וקטוריים הם הפורמליזם} – מבטיחים שנוכל להשתמש באלגברה ליניארית
\item \textbf{ממד גבוה = אתגר עצום} – קללת המימדיות דורשת ארכיטקטורות חכמות כמו \en{CNN}
\end{enumerate}

\textbf{מבט קדימה – פרק \num{2}:}

בפרק הבא נצלול לעומק \textbf{הדיכוטומיה של המידע} – מדוע יותר תכונות לא תמיד טוב יותר, ואיך להתמודד עם מרחבים בעלי ממד גבוה. נראה:

\begin{itemize}
\item הוכחה מתמטית מלאה של קללת המימדיות
\item השפעה על אלגוריתמים: \en{K-Nearest Neighbors, SVM, Decision Trees}
\item יחס פיצ'ר-דגימה: כמה נתונים באמת צריך?
\item פתרונות: \en{Feature Selection, PCA, Regularization}
\end{itemize}

\textbf{שאלת מחשבה לסיום:}

אם תמונת חתול $\num{1000} \times \num{1000}$ היא נקודה במרחב מיליון-ממדי, וכמעט כל "הזזה" קטנה היא אקסטרפולציה – איך מודלים גנרטיביים כמו \en{DALL-E} מצליחים ליצור תמונות חדשות שנראות הגיוניות?

התשובה תחכה לפרק \num{5} על קונוולוציה ורשתות \en{CNN}.

\subsection*{מטלות וקריאה מורחבת}

\textbf{תרגיל \num{1.1}:} מצאו מילים דומות באמצעות \en{Word2Vec} (ראו פסאודו-קוד).

\textbf{תרגיל \num{1.2}:} המחישו את קללת המימדיות (ראו פסאודו-קוד).

\textbf{תרגיל \num{1.3}:} חשבו ידנית מכפלה סקלרית, נורמה, ו\en{-}Cosine Similarity בין:
\[
\vec{u} = [\num{1}, \num{2}, \num{3}], \quad \vec{v} = [\num{4}, \num{5}, \num{6}]
\]

פתרון: $\langle \vec{u}, \vec{v} \rangle = \num{32}$, $\|\vec{u}\| = \sqrt{\num{14}}$, $\|\vec{v}\| = \sqrt{\num{77}}$, דמיון $= \num{0.975}$

\textbf{קריאה מורחבת:}

\begin{itemize}
\item \cite{strang2019} – \en{Linear Algebra and Learning from Data}, פרקים \num{1-2}: יסודות אלגברה ליניארית
\item \cite{mikolov2013} – המאמר המקורי על \en{Word2Vec}: \en{"Efficient Estimation of Word Representations in Vector Space"}
\item \cite{caliskan2017} – מחקר על הטיות במודלי שפה: \en{"Semantics Derived Automatically from Language Corpora Contain Human-like Biases"}
\item \cite{goodfellow2016} – \en{Deep Learning}, פרק \num{5}: למידת מכונה בסיסית
\end{itemize}

\textbf{שאלות להעמקה:}

\begin{enumerate}
\item מדוע \en{Cosine Similarity} עדיף על מרחק אוקלידי במרחבים רב-ממדיים?
\item האם ניתן להסיר לחלוטין הטיות ממודלי \en{Word2Vec}? מה המחיר?
\item איך רשתות \en{CNN} "מרמות" את קללת המימדיות?
\end{enumerate}

\textbf{סיום פרק \num{1}}