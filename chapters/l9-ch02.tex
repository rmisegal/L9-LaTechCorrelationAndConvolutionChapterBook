% l9-ch02.tex
% פרק 2: הדיכוטומיה של המידע – קללת המימדיות
% מחבר: ד"ר יורם סגל
% תאריך: ספטמבר 2025

\hebrewsection{הדיכוטומיה של המידע: קללת המימדיות}

\noindent\textbf{\en{The Information Dichotomy: Curse of Dimensionality}}

\vspace{0.5cm}

בפרק זה נחקור את אחד האתגרים המרכזיים של למידת מכונה: הפרדוקס שבו הוספת מידע (תכונות) עלולה \textit{להחליש} את המודל במקום לחזק אותו. נבחן את היסודות המתמטיים של קללת המימדיות, נוכיח את השפעותיה, ונראה כיצד היא משפיעה על אלגוריתמים שונים.

\hebrewsubsection{הפרדוקס: מתי "יותר" זה "פחות"?}

דמיינו רופא שמנסה לאבחן מחלה. בתחילה, יש לו שלוש תכונות: טמפרטורה, דופק, ולחץ דם. המודל שלו עובד היטב. לפתע, הוא מקבל גישה למעבדה מתקדמת שמודדת \num{1000} תכונות ביוכימיות. אינטואיטיבית, המידע הנוסף אמור לשפר את האבחנה. אך במציאות, דיוק המודל \textit{יורד}.

\textbf{למה זה קורה?}

התשובה טמונה בתופעה מתמטית מפתיעה שגילה \en{Richard Bellman} ב\en{-}\hebyear{1957} \cite{bellman1957}: ככל שמספר הממדים גדל, המרחב "מתנפח" באופן אקספוננציאלי, והנתונים הופכים דלילים \en{(Sparse)} – כמו כוכבים ביקום המתרחב.

\hebrewsubsection{בלמן והולדת המושג}

\en{Richard Ernest Bellman} (\hebyear{1920}–\hebyear{1984}) היה מתמטיקאי אמריקאי שתרם תרומות מהפכניות לתחומי תכנות דינמי, תורת הבקרה, ולמידת מכונה. בספרו \en{"Dynamic Programming"} (\hebyear{1957}), הוא זיהה בעיה מהותית: \textbf{מורכבות חישובית גדלה באופן אקספוננציאלי עם מספר המשתנים}.

\textbf{התובנה המרכזית של בלמן:}

במרחב חד-ממדי, אם רוצים לכסות קטע באורך \num{1} ברשת של נקודות במרווח $\epsilon$, נדרשות $\sim \num{1}/\epsilon$ נקודות. אך במרחב $d$-ממדי, נדרשות $\sim (\num{1}/\epsilon)^d$ נקודות – גידול אקספוננציאלי \cite{bellman1957}.

\textbf{דוגמה מספרית:}

\begin{hebrewtable}[H]
\caption{מספר הנקודות הנדרש לכיסוי מרחב \en{[0,1]} בצפיפות $\epsilon = \num{0.1}$}
\centering
\begin{rtltabular}{|m{3cm}|m{4cm}|m{2cm}|}
\hline
\textbf{\hebheader{סדר גודל}} & \textbf{\hebheader{נקודות נדרשות}} & \textbf{\hebheader{ממד}} \\
\hline
\hebcell{עשרות} & \encell{$\num{10}^1 = \num{10}$} & \encell{\num{1}} \\
\hline
\hebcell{מאות} & \encell{$\num{10}^2 = \num{100}$} & \encell{\num{2}} \\
\hline
\hebcell{אלפים} & \encell{$\num{10}^3 = \num{1000}$} & \encell{\num{3}} \\
\hline
\hebcell{עשרה מיליארד} & \encell{$\num{10}^{10}$} & \encell{\num{10}} \\
\hline
\hebcell{גוגול (יותר מאטומי היקום)} & \encell{$\num{10}^{100}$} & \encell{\num{100}} \\
\hline
\hebcell{בלתי נתפס} & \encell{$\num{10}^{1000000}$} & \encell{\num{1000000}} \\
\hline
\end{rtltabular}
\end{hebrewtable}

\textbf{מסקנה:} תמונת חתול $\num{1000} \times \num{1000}$ פיקסלים (מיליון ממדים) דורשת מספר דגימות שהוא \textit{מעבר ליכולת חישובית של כל מחשבי העולם ביחד}.

\hebrewsubsection{הוכחה מתמטית: התרחקות מהמרכז}

נוכיח תופעה מפתיעה: במרחבים רב-ממדיים, \textbf{כמעט כל הנפח מרוכז בקליפה החיצונית}, רחוק מהמרכז.

\textbf{משפט \num{2.1} – ריכוז הנפח בקליפה:}

יהי כדור יחידה $B_d = \{\vec{x} \in \mathbb{R}^d : \|\vec{x}\| \leq \num{1}\}$ במימד $d$. נפח הכדור הוא:

\begin{equation}
V_d(r) = \frac{\pi^{d/2}}{\Gamma(d/2 + \num{1})} r^d
\end{equation}

כאשר $\Gamma$ היא פונקציית הגמא, ו\en{-}$r$ הוא הרדיוס.

\textbf{הוכחה – יחס הנפחים:}

נחשב את יחס הנפח של הכדור הפנימי $B_d(\num{0.9})$ (רדיוס \num{0.9}) לכדור המלא $B_d(\num{1})$:

\begin{align}
\frac{V_d(\num{0.9})}{V_d(\num{1})} &= \frac{(\num{0.9})^d \cdot \pi^{d/2} / \Gamma(d/2 + \num{1})}{\num{1}^d \cdot \pi^{d/2} / \Gamma(d/2 + \num{1})} \nonumber \\
&= (\num{0.9})^d
\end{align}

\textbf{נחשב עבור ממדים שונים:}

\begin{hebrewtable}[H]
\caption{יחס הנפח הפנימי (\percent{90} מהרדיוס) לנפח המלא}
\centering
\begin{rtltabular}{|m{3cm}|m{3cm}|m{2.5cm}|}
\hline
\textbf{\hebheader{אחוז מהנפח}} & \textbf{\hebheader{$(\num{0.9})^d$}} & \textbf{\hebheader{ממד \en{d}}} \\
\hline
\encell{\percent{81}} & \encell{\num{0.81}} & \encell{\num{2}} \\
\hline
\encell{\percent{73}} & \encell{\num{0.729}} & \encell{\num{3}} \\
\hline
\encell{\percent{35}} & \encell{\num{0.349}} & \encell{\num{10}} \\
\hline
\encell{\percent{0.003}} & \encell{$\num{2.66} \times \num{10}^{-5}$} & \encell{\num{100}} \\
\hline
\hebcell{כמעט אפס} & \encell{$\sim \num{10}^{-46}$} & \encell{\num{1000}} \\
\hline
\end{rtltabular}
\end{hebrewtable}

\textbf{מסקנה מדהימה:} ב\en{-}\num{100} ממדים, רק \percent{0.003} מהנפח נמצא ב\en{-}\percent{90} המרכזיים! כמעט כל הנפח מרוכז בקליפה הדקה בין $r = \num{0.9}$ ל\en{-}$r = \num{1}$.

\textbf{משמעות ל\en{-}AI:}

אם דגימות האימון שלנו הן "נקודות במרכז", המודל שלנו יצטרך לעשות אקסטרפולציה עצומה כדי לחזות במרחב האמיתי (הקליפה החיצונית) שבו נמצאות רוב ה"תמונות האפשריות".

\hebrewsubsection{הוכחה: התכנסות המרחקים}

תופעה נוספת של קללת המימדיות: במרחבים רב-ממדיים, \textbf{כל המרחקים בין נקודות הופכים להיות דומים}.

\textbf{משפט \num{2.2} – התכנסות יחס המרחקים:}

יהיו $\{\vec{x}_i\}_{i=1}^n$ נקודות בלתי-תלויות במרחב $\mathbb{R}^d$. יהיו $d_{\max}$ ו\en{-}$d_{\min}$ המרחקים המקסימלי והמינימלי מנקודת מבחן $\vec{q}$. אזי:

\begin{equation}
\lim_{d \to \infty} \frac{d_{\max} - d_{\min}}{d_{\min}} \to \num{0}
\end{equation}

\textbf{הוכחה (סקיצה):}

\textit{צעד 1:} מרחק אוקלידי במימד $d$ בין $\vec{q}$ ל\en{-}$\vec{x}_i$:

\[
d_i = \|\vec{q} - \vec{x}_i\| = \sqrt{\sum_{j=1}^{d} (q_j - x_{ij})^2}
\]

\textit{צעד 2:} אם המרכיבים $(q_j - x_{ij})$ הם משתנים אקראיים בלתי-תלויים עם תוחלת $\mu$ ושונות $\sigma^2$, אזי לפי חוק המספרים הגדולים:

\[
\frac{\num{1}}{d}\sum_{j=1}^{d} (q_j - x_{ij})^2 \xrightarrow{P} \mu^2 + \sigma^2
\]

\textit{צעד 3:} לכן:

\[
d_i = \sqrt{d \cdot (\mu^2 + \sigma^2 + o(\num{1}))} = \sqrt{d} \cdot \sqrt{\mu^2 + \sigma^2} \cdot (1 + o(\num{1}))
\]

\textit{צעד 4:} כל המרחקים $d_i$ גדלים כ\en{-}$\sqrt{d}$, אך ההבדלים ביניהם גדלים לאט יותר (רק כ\en{-}$\sqrt{d \cdot \text{Var}}$). לכן:

\[
\frac{d_{\max} - d_{\min}}{d_{\min}} \sim \frac{O(\sqrt{d})}{ O(\sqrt{d})} \cdot \frac{\text{Var}}{\text{Mean}^2} \to \num{0} \quad \text{כאשר } d \to \infty
\]

$\blacksquare$

\textbf{משמעות מעשית:}

במרחב בעל \num{1000} ממדים, אם הנקודה הקרובה ביותר נמצאת במרחק \num{100}, הנקודה הרחוקה ביותר נמצאת במרחק $\sim \num{100.01}$ – \textbf{כמעט אותו דבר}. מדדי מרחק מאבדים יכולת הבחנה \cite{beyer1999, aggarwal2001}.

\hebrewsubsection{השפעה על אלגוריתמים: \en{K-Nearest Neighbors}}

אלגוריתם \en{K-Nearest Neighbors (KNN)}, שפותח על ידי \en{Evelyn Fix} ו\en{-}Joseph Hodges ב\en{-}\hebyear{1951} \cite{fix1951}, הוא אחד האלגוריתמים הפשוטים והאינטואיטיביים ביותר בלמידת מכונה.

\textbf{הרעיון:} כדי לסווג נקודה חדשה $\vec{q}$, מצא את $k$ הנקודות הקרובות ביותר במערך האימון, והצבע לפי רוב.

\textbf{למה \en{KNN} קורס במימד גבוה?}

מכיוון שכל המרחקים הופכים דומים (משפט \num{2.2}), \textbf{אין משמעות ל"קרוב ביותר"}. הנקודה ה\en{-}\num{1} הקרובה ביותר והנקודה ה\en{-}\num{1000} הקרובה ביותר נמצאות כמעט באותו מרחק!

\textbf{ניסוי מספרי – סימולציה:}

\begin{pythonbox}[הדגמת קריסת \en{KNN} במימד גבוה]
import numpy as np
from sklearn.neighbors import KNeighborsClassifier
from sklearn.datasets import make_classification
from sklearn.model_selection import train_test_split

def knn_curse_demo(dims, n_samples=1000):
    """
    Examines KNN performance as a function of number of dimensions.
    """
    results = []

    for d in dims:
        # Create synthetic data
        X, y = make_classification(
            n_samples=n_samples,
            n_features=d,
            n_informative=min(d, 10),  # Only 10 informative features
            n_redundant=0,
            random_state=42
        )

        X_train, X_test, y_train, y_test = train_test_split(
            X, y, test_size=0.3, random_state=42
        )

        # Train KNN
        knn = KNeighborsClassifier(n_neighbors=5)
        knn.fit(X_train, y_train)

        # Accuracy
        accuracy = knn.score(X_test, y_test)
        results.append(accuracy)

        print(f"Dim={d}: Accuracy={accuracy:.3f}")

    return results

# Run
dimensions = [2, 5, 10, 50, 100, 500, 1000]
accuracies = knn_curse_demo(dimensions)
\end{pythonbox}

\textbf{תוצאה צפויה:} הדיוק יורד מ\en{-}\percent{95} (במימד \num{2}) ל\en{-}\percent{55} (במימד \num{1000}) – כמעט אקראי.

\hebrewsubsection{יחס פיצ'ר-דגימה: כמה נתונים צריך?}

כפי שהדגיש ד"ר יורם סגל בהרצאתו, קיימים כללי אצבע לקביעת מספר הדגימות הנדרש:

\textbf{כלל \num{1} – למידת מכונה קלאסית:}

\begin{equation}
n_{\text{samples}} \geq \num{30} \times n_{\text{features}}
\end{equation}

\textbf{כלל \num{2} – למידה עמוקה:}

\begin{equation}
n_{\text{samples}} \geq \num{100} \times n_{\text{features}}
\end{equation}

\textbf{הצדקה מתמטית:}

כללים אלו נובעים מתורת הלמידה הסטטיסטית. לפי \textbf{גבול \en{VC (Vapnik-Chervonenkis)}} \cite{vapnik1971}, שגיאת ההכללה של מודל עם \en{VC-dimension} $d$ מוגבלת ע"י:

\begin{equation}
\epsilon \leq \sqrt{\frac{d \log(n/d) + \log(\num{1}/\delta)}{n}}
\end{equation}

כאשר $n$ הוא מספר הדגימות, $\delta$ רמת הביטחון, ו\en{-}$\epsilon$ שגיאת ההכללה.

\textbf{פירוש:} כדי לקבל שגיאה קטנה $\epsilon$, נדרש $n \sim O(d / \epsilon^2)$ – \textbf{מספר הדגימות צריך לגדול ליניארית עם המימד}.

\textbf{דוגמה מהחיים – תמונת חתול:}

תמונה $\num{1000} \times \num{1000}$ פיקסלים RGB:
\begin{itemize}
\item מספר תכונות: $\num{1000} \times \num{1000} \times \num{3} = \num{3000000}$
\item דגימות נדרשות (כלל \num{2}): $\num{100} \times \num{3000000} = \num{300000000}$ (שלוש מאות מיליון תמונות!)
\end{itemize}

\textbf{למה זה בלתי אפשרי?}

\begin{itemize}
\item \en{ImageNet}, אחד ממאגרי התמונות הגדולים בעולם, מכיל רק $\sim \num{1.2}$ מיליון תמונות אימון \cite{deng2009}
\item גם \en{Google} לא צילמה \num{300} מיליון תמונות חתולים
\end{itemize}

\textbf{אז איך \en{Deep Learning} מצליח?}

התשובה: \textbf{שיתוף משקלים והנחות אינדוקטיביות} \cite{lecun1998, goodfellow2016}.

\hebrewsubsection{הפתרון: רשתות \en{CNN} ומבנה מרחבי}

רשתות \en{Convolutional Neural Networks (CNN)}, שפותחו על ידי \en{Yann LeCun} ב\en{-}\hebyear{1989} \cite{lecun1989}, מתגברות על קללת המימדיות באמצעות שלושה עקרונות:

\textbf{עיקרון \num{1} – קישוריות מקומית} \en{(Local Connectivity)}:

במקום לחבר כל פיקסל לכל נוירון (שיצטרך $\num{1000}^2 \times \num{1000}^2 = \num{10}^{12}$ משקלים), כל נוירון מחובר רק ל\textbf{אזור קטן} (למשל, $\num{3} \times \num{3}$ פיקסלים).

\textbf{עיקרון \num{2} – שיתוף משקלים} \en{(Weight Sharing)}:

אותו פילטר (קרנל) משמש \textbf{בכל מקום בתמונה}. במקום $\num{10}^{12}$ משקלים, נדרשים רק $\num{3} \times \num{3} \times \num{64} = \num{576}$ משקלים לשכבה!

\textbf{עיקרון \num{3} – הירארכיה של תכונות}:

השכבות הראשונות לומדות תכונות פשוטות (קצוות, צבעים), והשכבות העמוקות לומדות תכונות מורכבות (עיניים, אוזניים, פנים).

\textbf{השוואת מספר הפרמטרים:}

\begin{hebrewtable}[H]
\caption{השוואת מספר פרמטרים: \en{Fully Connected} מול \en{CNN}}
\centering
\begin{rtltabular}{|m{5cm}|m{3.5cm}|m{3cm}|}
\hline
\textbf{\hebheader{דגימות נדרשות (כלל \num{2})}} & \textbf{\hebheader{מספר פרמטרים}} & \textbf{\hebheader{ארכיטקטורה}} \\
\hline
\hebcell{$\num{10}^{14}$ (בלתי אפשרי)} & \encell{$\sim \num{10}^{12}$} & \hebcell{\en{Fully Connected}} \\
\hline
\encell{$\num{6} \times \num{10}^9$} & \encell{$\sim \num{60} \times \num{10}^6$} & \hebcell{\en{CNN (AlexNet)}} \\
\hline
\encell{$\num{2.5} \times \num{10}^9$} & \encell{$\sim \num{25} \times \num{10}^6$} & \hebcell{\en{CNN (ResNet-50)}} \\
\hline
\end{rtltabular}
\end{hebrewtable}

\textbf{הצלחה מעשית:}

רשת \en{AlexNet} \cite{krizhevsky2012} אומנה על \num{1.2} מיליון תמונות בלבד (הרבה פחות מהנדרש תיאורטית), והשיגה פריצת דרך בתחרות \en{ImageNet} ב\en{-}\hebyear{2012}, עם הפחתה של \percent{10} בשגיאה לעומת השיטות הקודמות.

\hebrewsubsection{פתרונות נוספים: הפחתת ממדיות}

מלבד \en{CNN}, קיימות שיטות נוספות להתמודדות עם קללת המימדיות:

\textbf{שיטה \num{1} – \en{Feature Selection} (בחירת תכונות):}

בחירת תת-קבוצה של התכונות החשובות ביותר. שיטות נפוצות:
\begin{itemize}
\item \textbf{\en{Filter Methods}}: מיון לפי קורלציה, \en{mutual information}, \en{chi-square}
\item \textbf{\en{Wrapper Methods}}: בחירה לפי ביצועי המודל (\en{forward selection, backward elimination})
\item \textbf{\en{Embedded Methods}}: רגולריזציה \en{L1 (Lasso)} שמאלצת משקלים לאפס
\end{itemize}

\textbf{שיטה \num{2} – \en{PCA (Principal Component Analysis)}:}

המצאה של \en{Karl Pearson} (\hebyear{1901}) \cite{pearson1901} ו\en{-}Harold Hotelling (\hebyear{1933}) \cite{hotelling1933}. \en{PCA} מוצא את הכיוונים בעלי השונות המקסימלית ומקרין עליהם.

\textbf{רעיון \en{PCA}:}

אם $\mathbf{X} \in \mathbb{R}^{n \times d}$ היא מטריצת הנתונים, \en{PCA} מוצא את הווקטורים העצמיים של מטריצת הקווריאנס $\mathbf{C} = \frac{\num{1}}{n}\mathbf{X}^T\mathbf{X}$.

הווקטורים העצמיים $\{\vec{v}_1, \ldots, \vec{v}_d\}$ עם הערכים העצמיים הגדולים ביותר $\{\lambda_1 \geq \lambda_2 \geq \cdots \geq \lambda_d\}$ מהווים כיוונים של שונות מקסימלית.

\textbf{הקרנה על $k$ מרכיבים עיקריים:}

\begin{equation}
\mathbf{Z} = \mathbf{X} \mathbf{V}_k
\end{equation}

כאשר $\mathbf{V}_k = [\vec{v}_1, \ldots, \vec{v}_k]$ ו\en{-}$\mathbf{Z} \in \mathbb{R}^{n \times k}$ הוא הייצוג המופחת ($k \ll d$).

\textbf{כמה מרכיבים לבחור?}

כלל אצבע: בחר $k$ כך שה\textbf{שונות המוסברת} היא לפחות \percent{95}:

\begin{equation}
\frac{\sum_{i=1}^{k} \lambda_i}{\sum_{i=1}^{d} \lambda_i} \geq \num{0.95}
\end{equation}

\textbf{פסאודו-קוד – \en{PCA} ב\en{-}NumPy:}

\begin{pythonbox}[\en{PCA} – הפחתת ממדיות]
import numpy as np

def pca_manual(X, n_components=2):
    """
    Performs manual PCA using SVD.

    Args:
        X: data matrix (n_samples, n_features)
        n_components: number of principal components

    Returns:
        X_reduced: reduced data (n_samples, n_components)
        components: eigenvectors
        explained_var_ratio: explained variance ratio
    """
    # Center the data (subtract mean)
    X_centered = X - np.mean(X, axis=0)

    # Singular Value Decomposition
    # U: eigenvectors of XX^T
    # S: singular values (square root of eigenvalues)
    # Vt: eigenvectors of X^TX (principal components)
    U, S, Vt = np.linalg.svd(X_centered, full_matrices=False)

    # Select k principal components
    components = Vt[:n_components]

    # Projection
    X_reduced = X_centered @ components.T

    # Compute explained variance
    explained_variance = (S**2) / (len(X) - 1)
    explained_var_ratio = explained_variance[:n_components] / np.sum(explained_variance)

    print(f"Explained variance ratio: {explained_var_ratio}")
    print(f"Total: {np.sum(explained_var_ratio):.3f}")

    return X_reduced, components, explained_var_ratio

# Example
X = np.random.rand(1000, 100)  # 1000 samples, 100 features
X_reduced, components, var_ratio = pca_manual(X, n_components=10)
print(f"Original shape: {X.shape}, Reduced shape: {X_reduced.shape}")
\end{pythonbox}

\textbf{הערה על \en{SVD}:} \en{Singular Value Decomposition} הוא שיטה יעילה יותר לחישוב \en{PCA} מאשר חישוב ישיר של ערכים עצמיים, במיוחד כאשר $n \ll d$ \cite{golub1970}.

\hebrewsubsection{תרגיל תכנות עצמי \num{2.1} – הפחתת ממדיות עם \en{PCA}}

\textbf{מטרה:} להמחיש את יכולת \en{PCA} לשמר מידע תוך הפחתת ממדים.

\textbf{משימה:}

\begin{enumerate}
\item טענו את מערך הנתונים \en{MNIST} (תמונות ספרות כתובות ביד, $\num{28} \times \num{28}$ פיקסלים = \num{784} תכונות)
\item הפחיתו ל\en{-}\num{50} מרכיבים עיקריים באמצעות \en{PCA}
\item שחזרו תמונות מהייצוג המופחת
\item השוו את התמונות המקוריות למשוחזרות
\item חשבו את אחוז השונות המוסברת
\end{enumerate}

\textbf{פסאודו-קוד מורחב:}

\begin{pythonbox}[\en{PCA} על \en{MNIST} – דוגמה מלאה]
from sklearn.datasets import fetch_openml
from sklearn.decomposition import PCA
import matplotlib.pyplot as plt
import numpy as np

# Load MNIST
mnist = fetch_openml('mnist_784', version=1, parser='auto')
X = mnist.data.to_numpy()[:1000]  # First 1000 samples
y = mnist.target.to_numpy()[:1000]

# Normalization (important!)
X = X / 255.0

print(f"Original shape: {X.shape}")  # (1000, 784)

# PCA reduction to 50 components
pca = PCA(n_components=50)
X_reduced = pca.fit_transform(X)

print(f"Reduced shape: {X_reduced.shape}")  # (1000, 50)
print(f"Explained variance: {np.sum(pca.explained_variance_ratio_):.3f}")

# Reconstruct images
X_reconstructed = pca.inverse_transform(X_reduced)

# Display
fig, axes = plt.subplots(2, 5, figsize=(12, 5))
for i in range(5):
    # Original image
    axes[0, i].imshow(X[i].reshape(28, 28), cmap='gray')
    axes[0, i].set_title(f'Original: {y[i]}')
    axes[0, i].axis('off')

    # Reconstructed image
    axes[1, i].imshow(X_reconstructed[i].reshape(28, 28), cmap='gray')
    axes[1, i].set_title(f'Reconstructed')
    axes[1, i].axis('off')

plt.tight_layout()
plt.show()
\end{pythonbox}

\textbf{תוצאה צפויה:} עם \num{50} מרכיבים (במקום \num{784}), ניתן לשמר $\sim$\percent{85} מהמידע, והתמונות המשוחזרות יהיו קריאות למדי.

\hebrewsubsection{רגולריזציה: מניעת \en{Overfitting} במימד גבוה}

כאשר מספר התכונות גדול ביחס למספר הדגימות, המודל נוטה ל\textbf{התאמת יתר} \en{(Overfitting)} – הוא "לומד בעל-פה" את נתוני האימון ונכשל בנתונים חדשים.

\textbf{רגולריזציה} \en{(Regularization)} היא טכניקה שמוסיפה "קנס" על מורכבות המודל, ובכך מעודדת פשטות.

\textbf{שתי שיטות עיקריות:}

\textbf{רגולריזציה \en{L2 (Ridge)}:}

\begin{equation}
\mathcal{L}_{\text{Ridge}} = \sum_{i=1}^{n} (y_i - \hat{y}_i)^2 + \lambda \sum_{j=1}^{d} w_j^2
\end{equation}

קנס על \textbf{גודל המשקלים} – מעודד משקלים קטנים, אך לא מאפס אותם.

\textbf{רגולריזציה \en{L1 (Lasso)}:}

\begin{equation}
\mathcal{L}_{\text{Lasso}} = \sum_{i=1}^{n} (y_i - \hat{y}_i)^2 + \lambda \sum_{j=1}^{d} |w_j|
\end{equation}

קנס על \textbf{סכום ערכי המשקלים} – מאלץ חלק מהמשקלים להיות \textbf{בדיוק אפס}, ובכך מבצע בחירת תכונות אוטומטית.

\textbf{פרמטר הקנס} $\lambda$:
\begin{itemize}
\item $\lambda = \num{0}$: אין רגולריזציה (סיכון ל\en{-}Overfitting)
\item $\lambda$ גדול מאוד: המודל פשוט מדי (\en{Underfitting})
\item $\lambda$ אופטימלי: נקבע באמצעות \en{Cross-Validation}
\end{itemize}

\textbf{מי המציא?}

\en{Ridge}: \en{Arthur Hoerl} ו\en{-}Robert Kennard (\hebyear{1970}) \cite{hoerl1970}

\en{Lasso}: \en{Robert Tibshirani} (\hebyear{1996}) \cite{tibshirani1996}

\textbf{פסאודו-קוד – השוואת \en{Ridge} ו\en{-}Lasso:}

\begin{pythonbox}[רגולריזציה – \en{Ridge} מול \en{Lasso}]
from sklearn.linear_model import Ridge, Lasso
from sklearn.datasets import make_regression
from sklearn.model_selection import train_test_split
import numpy as np

# Create synthetic data: n << d (curse of dimensionality!)
X, y = make_regression(
    n_samples=100,      # Only 100 samples
    n_features=500,     # 500 features!
    n_informative=10,   # Only 10 relevant
    noise=10,
    random_state=42
)

X_train, X_test, y_train, y_test = train_test_split(
    X, y, test_size=0.3, random_state=42
)

# Ridge
ridge = Ridge(alpha=1.0)  # alpha = lambda
ridge.fit(X_train, y_train)
ridge_score = ridge.score(X_test, y_test)

# Lasso
lasso = Lasso(alpha=1.0)
lasso.fit(X_train, y_train)
lasso_score = lasso.score(X_test, y_test)

print(f"Ridge R²: {ridge_score:.3f}")
print(f"Lasso R²: {lasso_score:.3f}")

# How many Lasso weights are zero?
zero_weights = np.sum(np.abs(lasso.coef_) < 1e-5)
print(f"Lasso zeroed out {zero_weights}/{len(lasso.coef_)} features")
\end{pythonbox}

\textbf{תוצאה צפויה:} \en{Lasso} יאפס $\sim\num{490}$ מתוך \num{500} המשקלים, וישמור רק את התכונות האינפורמטיביות.

\hebrewsubsection{מחקר עדכני: אקסטרפולציה מול אינטרפולציה}

מחקר פורץ דרך של \en{Anadi Chaman} ו\en{-}Ivan Dokmanic (\hebyear{2021}) \cite{chaman2021} גילה תוצאה מדהימה: \textbf{במרחבים עם יותר מ\en{-}\num{100} ממדים, דגימה חדשה כמעט אף פעם אינה אינטרפולציה – היא כמעט תמיד אקסטרפולציה}.

\textbf{הגדרות:}

\begin{itemize}
\item \textbf{אינטרפולציה}: דגימה חדשה נמצאת \textit{בתוך} המעטפת הקמורה \en{(Convex Hull)} של נתוני האימון
\item \textbf{אקסטרפולציה}: דגימה חדשה נמצאת \textit{מחוץ} למעטפת הקמורה
\end{itemize}

\textbf{המעטפת הקמורה} \en{(Convex Hull)}:

\begin{equation}
\text{Conv}(\mathcal{X}) = \left\{ \sum_{i=1}^{n} \alpha_i \vec{x}_i \mid \vec{x}_i \in \mathcal{X}, \alpha_i \geq \num{0}, \sum_{i=1}^{n} \alpha_i = \num{1} \right\}
\end{equation}

\textbf{התוצאה המפתיעה:}

בניסויים על מערכי נתונים אמיתיים (\en{MNIST, CIFAR-10}), \en{Chaman} ו\en{-}Dokmanic מצאו ש\en{-}\percent{99.9} מדגימות הבדיקה הן אקסטרפולציה במרחב המקורי!

\textbf{משמעות ל\en{-}AI:}

כל פעם שמודל \en{Deep Learning} מסווג תמונה חדשה, הוא למעשה \textbf{משער מעבר לנתונים שראה}. ההצלחה של מודלים מודרניים נובעת מה\textbf{הנחות האינדוקטיביות} שלהם (כמו מבנה ה\en{-}CNN) ולא מכיסוי מלא של המרחב.

\textbf{שאלת מחשבה שהעלה המרצה:}

"אם כל תחזית היא אקסטרפולציה, האם \en{AI} באמת 'מבין' את העולם, או שהוא רק מנחש בצורה חכמה?"

זוהי אחת השאלות הפילוסופיות המרכזיות של \en{AI} מודרני.

\hebrewsubsection{סיכום ומבט קדימה}

\textbf{מה למדנו בפרק זה?}

\begin{enumerate}
\item \textbf{קללת המימדיות היא אמיתית ומוכחת} – נפח המרחב גדל אקספוננציאלית, והנתונים הופכים דלילים
\item \textbf{מרחקים מתכנסים} – במימד גבוה, כל הנקודות נעשות "באותו מרחק", ומדדים כמו \en{KNN} קורסים
\item \textbf{יחס פיצ'ר-דגימה הוא קריטי} – נדרשות לפחות \num{30-100} דגימות לכל תכונה
\item \textbf{רשתות \en{CNN} מתגברות על הקללה} – שיתוף משקלים וקישוריות מקומית מפחיתים דרמטית את מספר הפרמטרים
\item \textbf{הפחתת ממדיות עובדת} – \en{PCA, Feature Selection, Regularization} מאפשרים עבודה יעילה
\item \textbf{רוב התחזיות הן אקסטרפולציה} – \en{AI} מצליח בזכות הנחות אינדוקטיביות, לא כיסוי מלא
\end{enumerate}

\textbf{מבט קדימה – פרק \num{3}:}

בפרק הבא נעבור מהבעיות לפתרונות. נחקור את \textbf{מקדם הקביעה \Rsquared{}} – הכלי המרכזי להערכת מודלי רגרסיה. נראה:

\begin{itemize}
\item הגדרה מתמטית מלאה של \Rsquared{}
\item הוכחה: למה \Rsquared{} תמיד בין \num{0} ל\en{-}\num{1}?
\item סכנות: מתי \Rsquared{} מטעה?
\item \en{Adjusted \Rsquared{}} – הגרסה המתוקנת
\item קשר לקורלציה ולמכפלה סקלרית
\end{itemize}

\textbf{שאלת מחשבה לסיום:}

אם מודל רגרסיה ליניארית השיג \Rsquared{} = \num{0.95} על נתוני האימון, אך רק \Rsquared{} = \num{0.60} על נתוני הבדיקה – מה קרה? האם זו קללת המימדיות, \en{Overfitting}, או משהו אחר?

\subsection*{מטלות וקריאה מורחבת}

\textbf{תרגיל \num{2.1}:} הפחיתו את \en{MNIST} עם \en{PCA} (ראו פסאודו-קוד).

\textbf{תרגיל \num{2.2}:} השוו \en{Ridge} ו\en{-}Lasso על נתונים עם $n < d$ (ראו פסאודו-קוד).

\textbf{תרגיל \num{2.3}:} הוכיחו בעצמכם את משפט \num{2.1} (ריכוז הנפח בקליפה) עבור $d = \num{10}$.

רמז: חשבו $V_d(r) = \frac{\pi^{d/2}}{\Gamma(d/2 + \num{1})} r^d$ עבור $r = \num{0.9}$ ו\en{-}$r = \num{1}$.

\textbf{קריאה מורחבת:}

\begin{itemize}
\item \cite{bellman1957} – \en{Dynamic Programming}: המקור המקורי של "קללת המימדיות"
\item \cite{beyer1999} – \en{"When Is 'Nearest Neighbor' Meaningful?"}: ניתוח מתמטי של קריסת \en{KNN}
\item \cite{chaman2021} – \en{"Truly Generative or Just Extrapolation?"}: המחקר על אינטרפולציה מול אקסטרפולציה
\item \cite{goodfellow2016} – \en{Deep Learning}, פרק \num{7}: רגולריזציה והפחתת \en{Overfitting}
\item \cite{hastie2009} – \en{The Elements of Statistical Learning}, פרק \num{3}: רגרסיה ליניארית ורגולריזציה
\end{itemize}

\textbf{שאלות להעמקה:}

\begin{enumerate}
\item מדוע \en{Lasso} מאפס משקלים אך \en{Ridge} לא? רמז: חשבו על הגיאומטריה של הקנס ($L_1$ מול $L_2$)
\item האם קללת המימדיות משפיעה גם על רשתות \en{Transformer} ב\en{-}NLP?
\item אם \percent{99.9} מהתחזיות הן אקסטרפולציה, איך מסבירים את ההצלחה של \en{GPT-4}?
\end{enumerate}

\textbf{סיום פרק \num{2}}