% l9-ch03.tex
% פרק 3: הערכת מודלים – מקדם הקביעה \Rsquared{}
% מחבר: ד"ר יורם סגל
% תאריך: ספטמבר 2025

\hebrewsection{הערכת מודלים: מקדם הקביעה \Rsquared{}}

\noindent\textbf{\en{Model Evaluation: The Coefficient of Determination \Rsquared{}}}

\vspace{0.5cm}

בפרק זה נחקור את אחד הכלים המרכזיים להערכת מודלי רגרסיה: \textbf{מקדם הקביעה} \en{\Rsquared{} (R-squared)}. נבין את המשמעות המתמטית שלו, נוכיח מדוע הוא תמיד בין \num{0} ל\en{-}\num{1}, ונראה מתי הוא עלול להטעות אותנו.

\hebrewsubsection{השאלה המרכזית: מה זה מודל "טוב"?}

דמיינו שבניתם מודל רגרסיה ליניארית לחיזוי מחיר דירה על סמך שטח, מיקום וגיל. המודל נותן תחזיות, אך \textbf{כיצד נדע אם הוא טוב?}

שאלה זו מעסיקה סטטיסטיקאים ומדעני נתונים מאז ראשית המדע. התשובה המודרנית, שהתגבשה במאה ה\en{-}\num{20}, היא \textbf{מקדם הקביעה} – מספר יחיד שמסכם את "טיב ההתאמה" של המודל לנתונים.

\hebrewsubsection{היסטוריה: מי המציא את \Rsquared{}?}

המושג פותח בהדרגה על ידי מספר סטטיסטיקאים:

\textbf{\en{Francis Galton}} (\hebyear{1822}–\hebyear{1911}) היה הראשון לחקור קורלציה ורגרסיה. בעבודתו \en{"Regression towards Mediocrity in Hereditary Stature"} (\hebyear{1886}) \cite{galton1886}, הוא גילה שילדים של הורים גבוהים נוטים להיות נמוכים יותר מהוריהם (רגרסיה לממוצע).

\textbf{\en{Karl Pearson}} (\hebyear{1857}–\hebyear{1936}), תלמידו של \en{Galton}, פיתח את \textbf{מקדם הקורלציה} $r$ (\hebyear{1896}) \cite{pearson1896}, שהוא השורש של \Rsquared{} במקרה של רגרסיה פשוטה.

\textbf{\en{Ronald Fisher}} (\hebyear{1890}–\hebyear{1962}) הרחיב את המושג ל\textbf{רגרסיה מרובה} \en{(Multiple Regression)} וניתח את חלוקת השונות \cite{fisher1925}.

\textbf{\en{Sewall Wright}} (\hebyear{1889}–\hebyear{1988}) טבע את הסימון \Rsquared{} ב\en{-}\hebyear{1921} \cite{wright1921}.

\hebrewsubsection{הגדרה: מהו \Rsquared{}?}

\textbf{הגדרה \num{3.1} – מקדם הקביעה:}

\Rsquared{} הוא מדד סטטיסטי המצביע על \textbf{שיעור השונות במשתנה התלוי} $Y$ \textbf{שמוסברת על ידי המשתנים הבלתי תלויים} $X$ במודל רגרסיה.

\begin{equation}
R^2 = 1 - \frac{SS_{\text{res}}}{SS_{\text{tot}}}
\end{equation}

כאשר:

\begin{itemize}
\item $SS_{\text{res}}$ – \textbf{סכום ריבועי השאריות} \en{(Residual Sum of Squares)}
\item $SS_{\text{tot}}$ – \textbf{סכום הריבועים הכולל} \en{(Total Sum of Squares)}
\end{itemize}

\textbf{פירוט המשתנים:}

\begin{align}
SS_{\text{res}} &= \sum_{i=1}^{n} (y_i - \hat{y}_i)^2 \\
SS_{\text{tot}} &= \sum_{i=1}^{n} (y_i - \bar{y})^2
\end{align}

כאשר:
\begin{itemize}
\item $y_i$ – הערך האמיתי של התצפית ה\en{-}$i$
\item $\hat{y}_i$ – הערך החזוי על ידי המודל עבור התצפית ה\en{-}$i$
\item $\bar{y} = \frac{1}{n}\sum_{i=1}^{n} y_i$ – ממוצע הערכים האמיתיים
\end{itemize}

\hebrewsubsection{משמעות אינטואיטיבית: מהו "הסבר שונות"?}

כדי להבין את \Rsquared{}, נתחיל מ\textbf{מודל הבסיס} – המודל הפשוט ביותר האפשרי.

\textbf{מודל הבסיס (Baseline Model):} תחזה תמיד את הממוצע $\bar{y}$.

זהו המודל הטיפש ביותר – הוא מתעלם מכל התכונות $X$ ופשוט אומר "התשובה היא הממוצע". שגיאת מודל זה היא $SS_{\text{tot}}$ – \textbf{השונות הכוללת} בנתונים.

\textbf{המודל שלנו:} משתמש ב\en{-}$X$ כדי לחזות $\hat{y}_i$.

שגיאת המודל שלנו היא $SS_{\text{res}}$ – \textbf{השונות שנותרה} אחרי שהמודל עשה את עבודתו.

\textbf{הפרשנות:}

\begin{equation}
R^2 = 1 - \frac{\hebmath{שגיאת המודל שלנו}}{\hebmath{שגיאת מודל הבסיס}} = 1 - \frac{SS_{\text{res}}}{SS_{\text{tot}}}
\end{equation}

\begin{itemize}
\item אם $R^2 = \num{0}$: המודל שלנו לא טוב יותר מהממוצע (חסר תועלת!)
\item אם $R^2 = \num{1}$: המודל מושלם – אין שגיאות כלל
\item אם $R^2 = \num{0.8}$: המודל מסביר \percent{80} מהשונות
\end{itemize}

\textbf{דוגמה מספרית:}

נניח שאנחנו מנבאים מחירי דירות. הממוצע הוא \num{500000} ש"ח.

\begin{hebrewtable}[H]
\caption{דוגמה: חישוב \Rsquared{} למחירי דירות}
\centering
\begin{rtltabular}{|r|r|r|r|r|}
\hline
\textbf{\hebcell{תצפית}} & \textbf{\hebcell{$y_i$ (אמיתי)}} & \textbf{\hebcell{$\hat{y}_i$ (חזוי)}} & \textbf{\hebcell{$(y_i - \bar{y})^2$}} & \textbf{\hebcell{$(y_i - \hat{y}_i)^2$}} \\
\hline
\num{1} & \num{600000} & \num{580000} & \num{10000000000} & \num{400000000} \\
\hline
\num{2} & \num{450000} & \num{470000} & \num{2500000000} & \num{400000000} \\
\hline
\num{3} & \num{700000} & \num{680000} & \num{40000000000} & \num{400000000} \\
\hline
\textbf{\hebcell{סה"כ}} & & & \mixedcell{$SS_{\text{tot}} = \num{52500000000}$} & \mixedcell{$SS_{\text{res}} = \num{1200000000}$} \\
\hline
\end{rtltabular}
\end{hebrewtable}

\textbf{חישוב:}

\[
R^2 = 1 - \frac{\num{1200000000}}{\num{52500000000}} = 1 - \num{0.023} = \num{0.977}
\]

\textbf{פרשנות:} המודל מסביר \percent{97.7} מהשונות במחירים – מודל מצוין!

\hebrewsubsection{הוכחה: \Rsquared{} תמיד בין \num{0} ל\en{-}\num{1}}

\textbf{משפט \num{3.1} – תחום \Rsquared{}:}

עבור רגרסיה ליניארית עם איבר חופשי, מתקיים: $\num{0} \leq R^2 \leq \num{1}$.

\textbf{הוכחה:}

\textit{חלק א': $R^2 \leq \num{1}$}

נוכיח ש\en{-}$SS_{\text{res}} \leq SS_{\text{tot}}$.

\textit{צעד 1:} מודל הרגרסיה הליניארית ממזער את $SS_{\text{res}}$ על ידי פתרון:

\[
\min_{\vec{w}} \sum_{i=1}^{n} (y_i - \vec{w}^T \vec{x}_i)^2
\]

\textit{צעד 2:} מודל הבסיס (חיזוי הממוצע) הוא \textit{מקרה פרטי} של רגרסיה ליניארית עם $\vec{w} = \vec{0}$ (למעט איבר חופשי = $\bar{y}$).

\textit{צעד 3:} מכיוון שהרגרסיה ממזערת, היא לא יכולה להיות גרועה יותר ממודל הבסיס:

\[
SS_{\text{res}} \leq SS_{\text{tot}}
\]

\textit{צעד 4:} לכן:

\[
\frac{SS_{\text{res}}}{SS_{\text{tot}}} \leq \num{1} \quad \Rightarrow \quad 1 - \frac{SS_{\text{res}}}{SS_{\text{tot}}} \leq \num{1} \quad \Rightarrow \quad R^2 \leq \num{1} \quad \blacksquare
\]

\textit{חלק ב': $R^2 \geq \num{0}$}

זה נכון \textbf{רק לרגרסיה ליניארית עם איבר חופשי}. במקרים מיוחדים (כמו רגרסיה דרך הראשית), \Rsquared{} יכול להיות שלילי!

עבור רגרסיה רגילה, המודל תמיד יכול לחזות את הממוצע (במקרה הגרוע ביותר), כך ש\en{-}$SS_{\text{res}} \leq SS_{\text{tot}}$ ולכן $R^2 \geq \num{0}$.

$\blacksquare$

\hebrewsubsection{ייצוג גרפי: חלוקת השונות}

ניתן לחשוב על השונות הכוללת כ"פאי" שמתחלק לשניים:

\begin{equation}
\underbrace{SS_{\text{tot}}}_{\hebmath{שונות כוללת}} = \underbrace{SS_{\text{reg}}}_{\hebmath{שונות מוסברת}} + \underbrace{SS_{\text{res}}}_{\hebmath{שונות לא מוסברת}}
\end{equation}

כאשר $SS_{\text{reg}} = \sum_{i=1}^{n} (\hat{y}_i - \bar{y})^2$ – השונות שהמודל "לכד".

\textbf{קשר ל\en{-}\Rsquared{}:}

\begin{equation}
R^2 = 1 - \frac{\hebmath{שגיאת המודל שלנו}}{\hebmath{שגיאת מודל הבסיס}} = 1 - \frac{SS_{\text{res}}}{SS_{\text{tot}}}
\end{equation}

זו הגרסה החיובית של ההגדרה – במקום "אחד פחות שאריות", זה "שיעור המוסבר".

\hebrewsubsection{קשר למקדם הקורלציה}

במקרה של \textbf{רגרסיה ליניארית פשוטה} (משתנה בלתי תלוי אחד), מתקיים:

\begin{equation}
R^2 = r^2
\end{equation}

כאשר $r$ הוא \textbf{מקדם הקורלציה של פירסון} בין $X$ ו\en{-}$Y$:

\begin{equation}
r = \frac{\sum_{i=1}^{n} (x_i - \bar{x})(y_i - \bar{y})}{\sqrt{\sum_{i=1}^{n} (x_i - \bar{x})^2} \sqrt{\sum_{i=1}^{n} (y_i - \bar{y})^2}}
\end{equation}

\textbf{הוכחה (סקיצה):}

ברגרסיה פשוטה, $\hat{y}_i = a + bx_i$ כאשר:

\[
b = \frac{\sum (x_i - \bar{x})(y_i - \bar{y})}{\sum (x_i - \bar{x})^2} = r \cdot \frac{\sigma_y}{\sigma_x}
\]

ניתן להוכיח (באלגברה מייגעת אך ישירה) ש\en{-}$SS_{\text{reg}} = r^2 \cdot SS_{\text{tot}}$, ולכן $R^2 = r^2$.

$\blacksquare$

\textbf{משמעות:} ברגרסיה פשוטה, \Rsquared{} הוא \textit{ריבוע הקורלציה}. אם $r = \num{0.9}$ (קורלציה גבוהה), אזי $R^2 = \num{0.81}$.

\hebrewsubsection{סכנה \num{1}: \Rsquared{} עולה תמיד כשמוסיפים תכונות}

הבעיה הקריטית של \Rsquared{}: \textbf{הוא אף פעם לא יורד כשמוסיפים תכונות}, גם אם הן חסרות תועלת!

\textbf{משפט \num{3.2} – מונוטוניות \Rsquared{}:}

אם מוסיפים תכונה למודל רגרסיה, $R^2$ לא יורד (ובדרך כלל עולה).

\textbf{הוכחה:}

\textit{צעד 1:} נניח מודל עם $p$ תכונות מושג $R^2_p$. נוסיף תכונה נוספת (סה"כ $p+1$).

\textit{צעד 2:} המודל החדש יכול תמיד לבחור משקל $w_{p+1} = \num{0}$ לתכונה החדשה, ובכך להשיג את אותה שגיאה כמו המודל הישן.

\textit{צעד 3:} מכיוון שהרגרסיה \textit{ממזערת} את $SS_{\text{res}}$, היא תמצא משקלים טובים או שווים:

\[
SS_{\text{res}}^{(p+1)} \leq SS_{\text{res}}^{(p)}
\]

\textit{צעד 4:} לכן:

\[
R^2_{p+1} = 1 - \frac{SS_{\text{res}}^{(p+1)}}{SS_{\text{tot}}} \geq 1 - \frac{SS_{\text{res}}^{(p)}}{SS_{\text{tot}}} = R^2_p \quad \blacksquare
\]

\textbf{המשמעות המסוכנת:}

אם נוסיף \num{1000} תכונות אקראיות (רעש טהור), \Rsquared{} יעלה! זה יוביל אותנו לחשוב שהמודל השתפר, אך באמת הוא פשוט \textbf{התאים יתר} \en{(Overfitted)}.

\textbf{דוגמה מספרית – סימולציה:}

\begin{pythonbox*}[הדגמת בעיית \Rsquared{} עם תכונות אקראיות]
import numpy as np
from sklearn.linear_model import LinearRegression
from sklearn.metrics import r2_score

# Simple data: Y = 2X + noise
np.random.seed(42)
n = 100
X_real = np.random.rand(n, 1)
y = 2 * X_real.flatten() + np.random.randn(n) * 0.5

# Test R² as a function of number of random features
r2_scores = []

for num_random_features in [0, 1, 5, 10, 20, 50]:
    # Add random features (noise!)
    X_random = np.random.randn(n, num_random_features)
    X_combined = np.hstack([X_real, X_random])

    # Train
    model = LinearRegression()
    model.fit(X_combined, y)

    # R² on training data
    y_pred = model.predict(X_combined)
    r2 = r2_score(y, y_pred)
    r2_scores.append(r2)

    print(f"Num features: {1 + num_random_features}, R²: {r2:.4f}")

# Result: R² increases with each feature, even if it's noise!
\end{pythonbox*}

\textbf{תוצאה צפויה:}

\begin{hebrewtable}[H]
\caption{עליית \Rsquared{} עם הוספת תכונות אקראיות}
\centering
\begin{rtltabular}{|r|r|}
\hline
\textbf{\hebcell{מספר תכונות}} & \textbf{\hebcell{\Rsquared{} (אימון)}} \\
\hline
\num{1} & \num{0.850} \\
\hline
\num{2} & \num{0.852} \\
\hline
\num{6} & \num{0.870} \\
\hline
\num{11} & \num{0.895} \\
\hline
\num{21} & \num{0.930} \\
\hline
\num{51} & \num{0.975} \\
\hline
\end{rtltabular}
\end{hebrewtable}

\Rsquared{} עלה מ\en{-}\num{0.85} ל\en{-}\num{0.975} למרות שהתכונות הנוספות היו רעש טהור!

\hebrewsubsection{הפתרון: \en{Adjusted \Rsquared{}}}

\textbf{\en{Adjusted \Rsquared{}}} (מקדם הקביעה המתוקן) פותר את הבעיה על ידי הטלת \textbf{"קנס"} על הוספת תכונות.

\textbf{הגדרה \num{3.2} – \en{Adjusted \Rsquared{}}:}

\begin{equation}
R^2_{\text{adj}} = 1 - \frac{SS_{\text{res}} / (n - p - 1)}{SS_{\text{tot}} / (n - 1)}
\end{equation}

כאשר:
\begin{itemize}
\item $n$ – מספר התצפיות
\item $p$ – מספר התכונות (לא כולל איבר חופשי)
\end{itemize}

\textbf{ניסוח חלופי:}

\begin{equation}
R^2_{\text{adj}} = 1 - (1 - R^2) \cdot \frac{n - 1}{n - p - 1}
\end{equation}

\textbf{משמעות:}

המונה והמכנה מחולקים במספר \textbf{דרגות החופש} \en{(Degrees of Freedom)}. ככל שיש יותר תכונות $p$, דרגות החופש קטנות יותר, והקנס גדל.

\textbf{תכונות של \en{Adjusted \Rsquared{}}:}

\begin{itemize}
\item \textbf{יכול לרדת} כשמוסיפים תכונה חסרת תועלת
\item \textbf{יכול להיות שלילי} (אם המודל גרוע מאוד)
\item תמיד $R^2_{\text{adj}} \leq R^2$
\end{itemize}

\textbf{מתי להשתמש?}

\begin{itemize}
\item \textbf{\Rsquared{}}: להערכת התאמה על נתוני אימון בודדים
\item \textbf{\en{Adjusted \Rsquared{}}}: להשוואה בין מודלים עם מספר תכונות שונה
\end{itemize}

\textbf{חזרה על הסימולציה עם \en{Adjusted \Rsquared{}}:}

\begin{pythonbox*}[\en{Adjusted \Rsquared{}} – מונע עליה מלאכותית]
def adjusted_r2(r2, n, p):
    """
    Compute Adjusted R².

    Args:
        r2: regular R²
        n: number of samples
        p: number of features (excluding intercept)
    """
    return 1 - (1 - r2) * (n - 1) / (n - p - 1)

# Recalculation
for i, num_random in enumerate([0, 1, 5, 10, 20, 50]):
    r2 = r2_scores[i]
    n = 100
    p = 1 + num_random
    adj_r2 = adjusted_r2(r2, n, p)

    print(f"Features: {p}, R²: {r2:.4f}, Adj R²: {adj_r2:.4f}")
\end{pythonbox*}

\textbf{תוצאה צפויה:}

\begin{hebrewtable}[H]
\caption{השוואה: \Rsquared{} מול \en{Adjusted \Rsquared{}}}
\centering
\begin{rtltabular}{|r|r|r|}
\hline
\textbf{\hebcell{תכונות}} & \textbf{\hebcell{\Rsquared{}}} & \textbf{\hebcell{\en{Adjusted \Rsquared{}}}} \\
\hline
\num{1} & \num{0.850} & \num{0.849} \\
\hline
\num{2} & \num{0.852} & \num{0.849} \\
\hline
\num{6} & \num{0.870} & \num{0.862} \\
\hline
\num{11} & \num{0.895} & \num{0.880} \\
\hline
\num{21} & \num{0.930} & \num{0.900} \\
\hline
\num{51} & \num{0.975} & \num{0.895} \\
\hline
\end{rtltabular}
\end{hebrewtable}

\en{Adjusted \Rsquared{}} יורד כשמוסיפים תכונות חסרות תועלת – בדיוק כמו שצריך!

\hebrewsubsection{סכנה \num{2}: \Rsquared{} לא מתאים לסיווג}

\Rsquared{} מיועד ל\textbf{רגרסיה} (תחזית של ערך מספרי). עבור \textbf{סיווג} (תחזית של קטגוריה), הוא לא מתאים.

\textbf{למה?}

בסיווג, $y_i \in \{\num{0}, \num{1}\}$ אך $\hat{y}_i$ יכול להיות כל מספר (למשל, \num{0.73} = הסתברות לקטגוריה \num{1}). חישוב $SS_{\text{res}}$ לא משקף את דיוק הסיווג.

\textbf{מדדים נכונים לסיווג:}

\begin{hebrewtable}[H]
\caption{מדדי הערכה למשימות שונות}
\centering
\begin{rtltabular}{|l|l|p{6cm}|}
\hline
\textbf{\hebcell{משימה}} & \textbf{\hebcell{מדד}} & \textbf{\hebcell{הסבר}} \\
\hline
\hebcell{רגרסיה} & \en{\Rsquared{}, RMSE, MAE} & \hebcell{\Rsquared{} מודד שיעור שונות מוסברת} \\
\hline
\hebcell{סיווג} & \mixedcell{\en{Accuracy, Precision, Recall, F1}} & \hebcell{מודדים אחוז סיווגים נכונים} \\
\hline
\hebcell{סיווג (התפלגות)} & \en{Log-Loss, AUC-ROC} & \hebcell{מודדים איכות הסתברויות} \\
\hline
\end{rtltabular}
\end{hebrewtable}

\hebrewsubsection{קשר לאלגברה ליניארית: הקרנות}

ניתן לראות רגרסיה ליניארית כ\textbf{הטלה} של $\vec{y}$ על המרחב שנפרש על ידי עמודות המטריצה $\mathbf{X}$.

\textbf{נוסחת הרגרסיה במטריצות:}

\begin{equation}
\vec{\hat{y}} = \mathbf{X}(\mathbf{X}^T\mathbf{X})^{-1}\mathbf{X}^T\vec{y} = \mathbf{P}\vec{y}
\end{equation}

כאשר $\mathbf{P} = \mathbf{X}(\mathbf{X}^T\mathbf{X})^{-1}\mathbf{X}^T$ היא \textbf{מטריצת ההטלה} \en{(Projection Matrix)}.

\textbf{משמעות גיאומטרית של \Rsquared{}:}

\[
R^2 = \frac{\|\vec{\hat{y}} - \bar{y}\vec{1}\|^2}{\|\vec{y} - \bar{y}\vec{1}\|^2} = \cos^2(\theta)
\]

כאשר $\theta$ היא הזווית בין $\vec{y} - \bar{y}\vec{1}$ (הנתונים הממורכזים) ל\en{-}$\vec{\hat{y}} - \bar{y}\vec{1}$ (החיזוי הממורכז).

\textbf{פרשנות:} \Rsquared{} הוא ריבוע הקוסינוס של הזווית בין הנתונים לחיזוי במרחב הממורכז – \textbf{מדד גיאומטרי של התאמה}.

\hebrewsubsection{תרגיל תכנות עצמי \num{3.1} – חישוב \Rsquared{} ידני}

\textbf{מטרה:} להבין את המשמעות של \Rsquared{} על ידי חישוב ידני.

\textbf{משימה:}

\begin{enumerate}
\item צרו נתונים סינתטיים: $y = \num{3}x + \num{5} + \text{רעש}$
\item אמנו מודל רגרסיה ליניארית
\item חשבו את \Rsquared{} בשלוש דרכים:
\begin{itemize}
\item נוסחה ישירה: $1 - SS_{\text{res}}/SS_{\text{tot}}$
\item באמצעות קורלציה: $r^2$
\item באמצעות \en{sklearn}
\end{itemize}
\item השוו את התוצאות (צריכות להיות זהות!)
\end{enumerate}

\textbf{פסאודו-קוד מלא:}

\begin{pythonbox*}[חישוב \Rsquared{} בשלוש דרכים]
from sklearn.metrics import r2_score

# Method 1: Direct formula
SS_tot = np.sum((y - np.mean(y))**2)
SS_res = np.sum((y - y_pred)**2)
r2_manual = 1 - (SS_res / SS_tot)

# Method 2: Correlation squared (simple regression)
r2_from_corr = np.corrcoef(X.flatten(), y)[0, 1]**2

# Method 3: sklearn
r2_sklearn = r2_score(y, y_pred)
\end{pythonbox*}

\textbf{תוצאה צפויה:} כל שלוש השיטות יתנו $R^2 \approx \num{0.95}$ (תלוי ברעש).

\hebrewsubsection{תרגיל תכנות עצמי \num{3.2} – השוואת \Rsquared{} ו\en{-}Adjusted \Rsquared{}}

\textbf{מטרה:} להמחיש את ההבדל בין \Rsquared{} ל\en{-}Adjusted \Rsquared{} כאשר מוסיפים תכונות.

\textbf{משימה:}

\begin{enumerate}
\item צרו נתונים עם תכונה אחת רלוונטית
\item הוסיפו בהדרגה תכונות אקראיות (רעש)
\item עקבו אחרי \Rsquared{} ו\en{-}Adjusted \Rsquared{} כפונקציה של מספר התכונות
\item הציגו גרף: ציר X = מספר תכונות, ציר Y = ציון
\end{enumerate}

\begin{pythonbox*}[\Rsquared{} מול \en{Adjusted \Rsquared{}} – השוואה]
def adjusted_r2(r2, n, p):
    return 1 - (1 - r2) * (n - 1) / (n - p - 1)

# As features increase: R² rises, Adjusted R² falls
for p in range(1, 51):
    adj_r2 = adjusted_r2(r2, 100, p)
\end{pythonbox*}

\textbf{תוצאה צפויה:} \Rsquared{} עולה בהתמדה, אך \en{Adjusted \Rsquared{}} יורד אחרי כמה תכונות.

\hebrewsubsection{מקרי קצה: מתי \Rsquared{} מטעה?}

\textbf{מקרה \num{1}: יחסים לא-ליניאריים}

אם הקשר בין $X$ ל\en{-}$Y$ הוא לא-ליניארי (למשל, ריבועי), רגרסיה ליניארית תיכשל ו\en{-}\Rsquared{} יהיה נמוך – אך זה לא אומר שאין קשר!

\textbf{דוגמה:} $y = x^2$

רגרסיה ליניארית תיתן \Rsquared{} נמוך, אך הקשר מושלם (רק לא ליניארי).

\textbf{פתרון:} השתמש ב\textbf{פולינומים} או מודלים לא-ליניאריים (כמו רשתות נוירונים).

\textbf{מקרה \num{2}: \en{Outliers} (ערכי חריגים)}

ערך חריג אחד יכול להוריד דרמטית את \Rsquared{}, גם אם המודל טוב ל\en{-}\percent{99} מהנתונים.

\textbf{פתרון:} זהה וטפל ב\en{-}Outliers לפני אימון (באמצעות \en{Z-score, IQR}, או שיטות עמידות כמו \en{RANSAC}).

\textbf{מקרה \num{3}: מתאם אינו סיבתיות}

\Rsquared{} גבוה לא אומר ש\en{-}$X$ \textit{גורם} ל\en{-}$Y$. ייתכן שיש משתנה שלישי $Z$ שגורם לשניהם.

\textbf{דוגמה קלאסית:} צריכת גלידה וטביעות מים מתואמות – לא כי גלידה גורמת לטביעות, אלא כי \textbf{קיץ} גורם לשניהם.

\hebrewsubsection{סיכום ומבט קדימה}

\textbf{מה למדנו בפרק זה?}

\begin{enumerate}
\item \textbf{\Rsquared{} מודד שיעור שונות מוסברת} – מספר יחיד שמסכם את איכות המודל
\item \textbf{תחום: \num{0} עד \num{1}} – הוכחנו מתמטית למה זה תמיד נכון (לרגרסיה עם \en{intercept})
\item \textbf{קשר לקורלציה} – ברגרסיה פשוטה, $R^2 = r^2$
\item \textbf{בעיה: עולה תמיד} – הוספת תכונות מעלה את \Rsquared{} גם אם הן חסרות תועלת
\item \textbf{פתרון: \en{Adjusted \Rsquared{}}} – מנרמל לפי מספר התכונות ומונע \en{Overfitting}
\item \textbf{לא לסיווג!} – \Rsquared{} מיועד לרגרסיה, לא לסיווג
\item \textbf{מגבלות} – מתאם אינו סיבתיות, רגיש ל\en{-}Outliers, מניח ליניאריות
\end{enumerate}

\textbf{מבט קדימה – פרק \num{4}:}

בפרק הבא נעבור מהערכת מודלים למדידת \textbf{קשרים בין משתנים}. נחקור:

\begin{itemize}
\item \textbf{קו-ווריאנס} \en{(Covariance)} – מדד גולמי של קשר ליניארי
\item \textbf{קורלציה} \en{(Correlation)} – הגרסה המנורמלת של קו-ווריאנס
\item \textbf{ייצוג אלגברי ליניארי} – קו-ווריאנס כמכפלה סקלרית, קורלציה כ\en{-}Cosine Similarity
\item \textbf{סכנות} – \en{"Correlation is not Causation"}, קורלציות מזויפות
\item \textbf{מטריצת קורלציה} – כלי לזיהוי \en{Multicollinearity}
\end{itemize}

\textbf{שאלת מחשבה לסיום:}

אם שני משתנים $X$ ו\en{-}$Y$ בעלי קורלציה $r = \num{0}$ (אורתוגונליים), האם זה אומר שהם בלתי תלויים סטטיסטית?

רמז: התשובה היא \textbf{לא}! קורלציה אפס משמעה רק \textit{אין קשר ליניארי}, אך עדיין יכול להיות קשר לא-ליניארי (למשל, $Y = X^2$).

\subsection*{מטלות וקריאה מורחבת}

\textbf{תרגיל \num{3.1}:} חשבו \Rsquared{} ידנית בשלוש דרכים (ראו פסאודו-קוד).

\textbf{תרגיל \num{3.2}:} השוו \Rsquared{} ו\en{-}Adjusted \Rsquared{} עם תכונות אקראיות (ראו פסאודו-קוד).

\textbf{תרגיל \num{3.3}:} הוכיחו בעצמכם שברגרסיה פשוטה $R^2 = r^2$.

רמז: התחילו מהביטוי $SS_{\text{reg}} = \sum (\hat{y}_i - \bar{y})^2$ והשתמשו ב\en{-}$\hat{y}_i = a + bx_i$.

\textbf{קריאה מורחבת:}

\begin{itemize}
\item \cite{galton1886} – \en{"Regression towards Mediocrity in Hereditary Stature"}: המאמר המקורי של \en{Galton}
\item \cite{pearson1896} – \en{"Mathematical Contributions to the Theory of Evolution"}: \en{Pearson} על קורלציה
\item \cite{fisher1925} – \en{"Statistical Methods for Research Workers"}: \en{Fisher} על ניתוח שונות
\item \cite{wright1921} – \en{"Correlation and Causation"}: \en{Wright} טובע את \Rsquared{}
\item \cite{hastie2009} – \en{The Elements of Statistical Learning}, פרק \num{3}: רגרסיה ליניארית ברזולוציה גבוהה
\end{itemize}

\textbf{שאלות להעמקה:}

\begin{enumerate}
\item למה \en{Adjusted \Rsquared{}} יכול להיות שלילי, אך \Rsquared{} לא?
\item מה יקרה ל\en{-}\Rsquared{} אם ננרמל את $X$ ו\en{-}$Y$ (נחסיר ממוצע ונחלק בסטיית תקן)?
\item איך \Rsquared{} מתייחס למודלים לא-ליניאריים (כמו רשתות נוירונים)?
\end{enumerate}

\textbf{סיום פרק \num{3}}